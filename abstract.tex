%\phantomsection
%\addcontentsline{toc}{section}{Abstract}
%\noindent \section*{Abstract}
\thispagestyle{empty}
\vspace*{-0.5in}
\hspace*{-0.5in}
\fboxsep=10pt\relax\fboxrule=4pt\relax
\specialbox{bgLightGray!30!white}{azure5}{75pt}{1.1\textwidth}{
{
\color{azure5}
\Large
\textbf{Résumé}\\\\
}
\small
Le présent rapport synthétise le travail effectué sur une période de six mois au sein de l'entreprise Cegedim SRH, et qui s'inscrit dans le cadre de la validation du projet de fin d'études du Master Ingénierie de Données et Développement Logiciel à la Faculté des Sciences de Rabat.\\\\
La première phase du travail a consisté à auditer et analyser l'architecture d'un portail de gestion électronique de documents appelé \textbf{Arkevia}, puis à identifier et corriger les problèmes et anomalies détectés, ainsi qu'à proposer des axes d'amélioration suite à l'audit architectural réalisé. Ensuite, dans une seconde phase, les travaux ont porté sur le mécanisme de notification chargé d'informer les utilisateurs des documents récemment déposés dans leurs coffres-forts Arkevia. En effet, étant donné l’importance que porte Cegedim à la qualité de ses solutions, il a été convenu de refondre ce mécanisme et de le rendre indépendant de l'application mère Arkevia. L'application produite est principalement basée sur une approche de traitement par lots multithread via les technologies Spring Boot et Spring Batch.\\\\
\normalsize
\textbf{Mot clés :} spring, batch, processing, sirh, ged.}
\par\vspace*{0.75in}
\hspace*{-0.5in}
\specialbox{bgLightGray!30!white}{actBlue}{75pt}{1.1\textwidth}{
    {
    \color{actBlue}
    \Large
    \textbf{Abstract}\\\\
    }
    \small
    This report summarizes the work carried out over a period of six months at Cegedim SRH, in the context of the approval of the end-of-study project for the Master's degree in Data Engineering and Software Development at the Faculty of Science in Rabat.\\\\
    The first phase of the work involved auditing and analyzing the architecture of an electronic document management portal called \textbf{Arkevia}, then identifying and repairing detected problems and anomalies, as well as suggesting new areas of enhancement based on the architectural review. In a second phase, the work focused on the notification mechanism responsible for informing users of documents recently deposited into their Arkevia safes. Indeed, given the importance that Cegedim places on the quality of their solutions, it was agreed to overhaul this mechanism and make it independent of the Arkevia parent application. The resulting application is primarily based on a multi-threaded batch approach using Spring Boot and Spring Batch technologies.\\\\
    \normalsize
    \textbf{Keywords :} spring, batch, processing, sirh, ged.
}

\newpage
