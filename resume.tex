\phantomsection
\addcontentsline{toc}{section}{\abstractname}
\noindent \section*{\abstractname}
	Le présent rapport synthétise le travail effectué sur une période de six mois au sein de l'entreprise Cegedim SRH, et qui s'inscrit dans le cadre de la validation du projet de fin d'études du Master Ingénierie de Données et Développement Logiciel à la Faculté des Sciences de Rabat.\\
	
	La première phase du travail a consisté à auditer et analyser l'architecture d'un portail de gestion électronique de documents appelé \textbf{Arkevia}, puis à identifier et corriger les problèmes et anomalies détectés, ainsi qu'à proposer des axes d'amélioration suite à l'audit architectural réalisé. Ensuite, dans une seconde phase, les travaux ont porté sur le mécanisme de notification chargé d'informer les utilisateurs des documents récemment déposés dans leurs coffres-forts Arkevia. En effet, étant donné l’importance que porte Cegedim à la qualité de ses solutions, il a été convenu de refondre ce mécanisme et de le rendre indépendant de l'application mère Arkevia. L'application produite est principalement basée sur une approche de traitement par lots multithread via les technologies Spring Boot et Spring Batch.
	\\\\
	\textbf{Mot clés :}\\
	spring, batch, processing, sirh, ged.
	\newpage