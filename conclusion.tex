%%%%%%%%%%%%%%%%%%%%%%%%%%%%%%%%%%%%%%%%%%%%%%%%%%%
%% Conclusion
%%%%%%%%%%%%%%%%%%%%%%%%%%%%%%%%%%%%%%%%%%%%%%%%%%%

\phantomsection
\addcontentsline{toc}{section}{Conclusion générale}
\section*{Conclusion générale}

Dans l'ensemble, je retiens un bilan positif de mon stage de fin d'études : J'ai pu intégrer une équipe de recherche et développement (R\&D) polyvalente travaillant principalement autour des technologies Java et JavaScript. Cela m'a donné une idée concrète des exigences de Cegedim en matière d'ingénierie de développement orienté objet,
ainsi que sur les règles et les bonnes pratiques de la gestion de projet agile, allant de la phase de recueil des besoins jusqu'à celle de la livraison du produit final.\\

La formation acquise dans le cadre de mon Master à la Faculté des Sciences de Rabat, notamment le module d'ingénierie du développement logiciel, m'a été très utile dans diverses situations que ce soit sur le plan conceptuel (élaboration des besoins, étude de faisabilité, conception générale, etc.) ou sur le plan technique (codage, complexité du code, bonnes pratiques de développement, etc.) ou encore sur le plan relationnel (travail collaboratif, planification et gestion de projets, etc.).\\
Ce stage a été très enrichissant pour moi, car il m'a permis de découvrir le monde du SIRH, un domaine vaste, nécessitant à la fois un esprit d'analyse sceptique et de solides compétences techniques. De surcroît, il m'a permis de participer concrètement à ses enjeux à travers des missions de développement et de refonte.