%%%%%%%%%%%%%%%%%%%%%%%%%%%%%%%%%%%%%%%%%%%%%%%%%%%
%% Conclusion
%%%%%%%%%%%%%%%%%%%%%%%%%%%%%%%%%%%%%%%%%%%%%%%%%%%

\phantomsection
\markboth{CONCLUSION GÉNÉRALE}{}
\addcontentsline{toc}{section}{Conclusion générale}
\section*{Conclusion générale}

Dans l'ensemble, je retiens un bilan positif de mon stage de fin d'études : J'ai pu intégrer une équipe de recherche et développement (R\&D) polyvalente travaillant principalement autour des technologies Java et JavaScript. Cela m'a donné une idée concrète des exigences de Cegedim en matière d'ingénierie de développement orienté objet, ainsi que sur les règles et les bonnes pratiques de la gestion de projet agile, allant de la phase de recueil des besoins jusqu'à celle de la livraison du produit final.\\

La formation acquise dans le cadre de mon Master à la Faculté des Sciences de Rabat, notamment le module d'ingénierie du développement logiciel, m'a été très utile dans diverses situations que ce soit sur le plan conceptuel (élaboration des besoins, étude de faisabilité, conception générale, etc.) ou sur le plan technique (codage, complexité du code, bonnes pratiques de développement, etc.) ou encore sur le plan relationnel (travail collaboratif, planification et gestion de projets, etc.).\\

Ce stage a été très enrichissant pour moi, car il m'a permis de découvrir le monde du SIRH, un domaine vaste, nécessitant à la fois un esprit d'analyse sceptique et de solides compétences techniques. De surcroît, cette expérience m'a permis de participer concrètement à ses enjeux à travers des missions de développement et de refonte.\\

Ce rapport présente les différentes tâches réalisées durant ma période de stage, à savoir la migration du socle technique et le refactoring du code ainsi que la refonte du système délégué d'envoi des notifications. L'élaboration de ce rapport a été entamé par une présentation de l'organisme d'accueil. Par la suite, une étude de l'existant a été réalisée afin d'identifier les problèmes et d'exposer les solutions et les axes d'amélioration. Après cela, une étude conceptuelle détaillée des différentes fonctionnalités a été mise en place de même que la définition des besoins et les contraintes du projet. Enfin, ce rapport se termine par l'exposition des outils et de l'environnement de travail et aussi par une démonstration sur une solution réalisée pour remédier aux inconvénients de l'ancien système d'envoi de notifications.\\

Au terme de ce mémoire, je pense que ce stage en entreprise m’a offert une bonne préparation pour mon insertion professionnelle, car il fut une expérience enrichissante et complète qui conforte le désir d’exercer un futur métier dans le domaine de l’informatique. D'autre part, l'environnement professionnel dans lequel s'est déroulé ce stage m'a fait découvrir de nouvelles connaissances et compétences et a renforcé mon esprit d'équipe et mon professionnalisme. Cette expérience a aiguisé mes capacités d'analyse et de synthèse et a surtout renforcé mon esprit critique, ma motivation et mon ambition.\\
Enfin, je tiens à exprimer ma satisfaction d’avoir pu travailler dans de bonnes conditions matérielles et un environnement agréable.