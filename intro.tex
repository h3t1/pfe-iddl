%%%%%%%%%%%%%%%%%%%%%%%%%%%%%%%%%%%%%%%%%%%%%%%%%%%
%% Introduction
%%%%%%%%%%%%%%%%%%%%%%%%%%%%%%%%%%%%%%%%%%%%%%%%%%%
\phantomsection
\addcontentsline{toc}{section}{Introduction générale}
\noindent \section*{Introduction générale}
\markboth{INTRODUCTION GÉNÉRALE}{}

\phantomsection
\addcontentsline{toc}{subsection}{Contexte du projet}
\subsection*{Contexte du projet}
Face à la croissance de l'activité de Cegedim SRH, et dans l’optique de garantir un produit de haute qualité, cette dernière a décidé d’adopter une stratégie d’amélioration et d’évolution des systèmes existants.\\

L'un des sujets qualifiés au titre des priorités fut la refonte d'un portail de gestion électronique de documents appelé Arkevia, également connu sous le nom de coffre-fort électronique salarié, un outil de réception et de stockage sécurisé dans lequel le salarié peut stocker ses documents professionnels et personnels : bulletins de paie, ou tout autre document importé au format dématérialisé, pièces d'identités, diplômes, factures, ou autres documents personnels. Ce portail a subi plusieurs changements et évolutions au cours de ces dernières années, rendant l'application volumineuse et difficile à gérer et à comprendre. Malgré les efforts déployés pour maintenir un code modulaire et évolutif, cela n'a pas été suffisant pour réduire la complexité du produit ni à virer les pratiques et méthodes obsolètes ni à corriger les problèmes de performances qui ne cessaient d'augmenter de façon spectaculaire avec le nombre croissant d'utilisateurs.\\

Conscient de ces enjeux, le département R\&D de Cegedim SRH de Rabat a opté pour s'engager dans la refonte de ce portail pour traiter les problèmes et les axes d’amélioration identifiés suite à l’audit architectural réalisé. C'est dans ce contexte que le département R\&D m'a confié ce sujet dont la mission principale est la refonte du portail Arkevia.
\addcontentsline{toc}{subsection}{Objectifs et missions}
\subsection*{Objectifs et missions}
Le stage est axé sur le développement de la refonte de la structure du portail Arkevia afin de répondre aux besoins suivants :
\begin{itemize}
    \item Analyse de l’existant et formalisation du besoin ;
    \item Revue du cœur de l’application et migration vers de nouvelles technologies ;
    \item Restructuration du projet et du mécanisme d'envoi de notifications ;
    \item Réalisation de nouvelles fonctionnalités ;
    \item Adaptation au processus de livraison ;
    \item Réalisation de tests unitaires ;
    \item Rédaction de rapports techniques sur l’avancement du sujet.
\end{itemize}

\addcontentsline{toc}{subsection}{Contenu du mémoire}
\subsection*{Contenu du mémoire}
Ce travail a abouti à la rédaction du présent mémoire, il est composé de cinq chapitres : 
\begin{itemize}
    \item \textbf{Le premier chapitre} est consacré à la présentation de l'organisation d'accueil, de ses domaines d'activité et de sa hiérarchie.
    \item \textbf{Le deuxième chapitre} constitue une étude des aspects du système existant, de son architecture globale et de son fonctionnement, permettant de révéler de manière synthétique le contexte général du projet, d'identifier les problèmes et de proposer des solutions et des axes d'amélioration.
    \item \textbf{Le troisième chapitre} couvrira les méthodes et pratiques employées pour la conduite de projets et l'organisation des tâches au sein de l'organisation d'accueil.
    \item \textbf{Le quatrième chapitre} est consacré à la conception détaillée du portail Arkevia (le portail existant), ainsi qu'à l'élaboration des besoins, des contraintes et des solutions requises.
    \item \textbf{Le cinquième chapitre} sera consacré à la présentation des différents outils et environnements de travail sur lesquels j'ai exercé pendant toute la période du stage, puis dans une seconde partie, à une démonstration d'un des travaux confiés à ma charge, à savoir le système d'envoi de notifications.
\end{itemize}